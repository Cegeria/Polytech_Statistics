\documentclass{article}
\usepackage{blindtext}
\usepackage{geometry}
 \geometry{
 a4paper,
 total={170mm,257mm},
 left=20mm,
 top=20mm,
 }
\usepackage{amsmath}
\usepackage{fdsymbol} %красивые символы
\usepackage{tikz-cd}
\usepackage{listings}
\tikzcdset{arrow style=math font}%диаграммы 
\usepackage{comment}
\usepackage{hyperref}

%для русского:
\usepackage[T2A]{fontenc}
\usepackage[utf8]{inputenc}
\usepackage[russian]{babel} 
\def\ZZ{\mathbb{Z}}
\def\NN{\mathbb{N}}

\begin{document}
\begin{titlepage}	

	\begin{center}		

		Санкт-Петербургский Государственный Политехнический университет\\
		Высшая школа прикладной математики и вычислительной физики\\
		Дисциплина "Математическая статистика"\\[6cm]

		\huge {Отчёт по лабораторной работе №1} \\[6cm]


	\end{center} 
	\begin{flushright} 
		\begin{minipage}{0.3\textwidth} 
			\begin{flushleft} 

				\textbf{Работу выполнил:}\\
				{Крупица С.В.}\\
				{Группа:  5030102/20101} \\

				\textbf{Преподаватель:}\\
				{Баженов А.Н.}\\

			\end{flushleft}
		\end{minipage}
	\end{flushright}

	\vfill
	\begin{center}
		\large{Санкт-Петербург} \\
		\large{2025}

\end{titlepage}
\newpage

\tableofcontents
\newpage

\section{\Large Формулировка задания}
Для 4 распределений:
\begin{itemize}
    \item Нормальное распределение N(x,0,1)
    \item Распределение Коши C(x,0,1)
    \item Распределение Пуассона P(k,1,0)
    \item Равномерное распределение U(x,$-\sqrt{3}$,$\sqrt{3}$)
\end{itemize}
\ \\
Задания:
\begin{enumerate}
    \item  Сгенерировать выборки размером 10, 50 и 1000 элементов.
 Построить на одном рисунке гистограмму и график плотности рас
пределения.
    \item Сгенерировать выборки размером 10, 100 и 1000 элементов.
 Для каждой выборки вычислить следующие статистические характе ристики положения данных: $\hat{x}, med x, z_Q$. 
 Повторить такие вычисления 1000 раз для каждой выборки и найти среднее характеристик положения и их квадратов, вычислить оценку дисперсии, представить полученные данные в виде таблицы.
\end{enumerate}

\section{\Large Ссылка на GitHub}


\url{https://github.com/Cegeria/Polytech_Statistics/new/main/lab_1}

\section{Теоретические сведения}

\subsection{Распределения}

\paragraph{Нормальное распределение \( N(0,1) \):}
Плотность распределения:
\[
  f(x) = \frac{1}{\sqrt{2\pi}}\, e^{-\frac{x^2}{2}}.
\]
Матожидание: \( M(X)=0 \), дисперсия: \( D(X)=1 \).

\paragraph{Распределение Коши \( C(0,1) \):}
Плотность распределения:
\[
  f(x) = \frac{1}{\pi}\, \frac{1}{1+x^2}.
\]
У распределения Коши не существуют математическое ожидание и дисперсия (в классическом смысле) из-за тяжёлых хвостов.

\paragraph{Распределение Пуассона \( P(\lambda=10) \):}
Функция вероятностей (pmf):
\[
  P(X=k) = \frac{10^k\, e^{-10}}{k!}, \quad k = 0,1,2,\dots.
\]
Матожидание: \( M(X)=10 \), дисперсия: \( D(X)=10 \).

\paragraph{Равномерное распределение \( U(-\sqrt{3},\sqrt{3}) \):}
Плотность распределения:
\[
  f(x) = 
  \begin{cases}
    \dfrac{1}{2\sqrt{3}}, & x \in \left[-\sqrt{3},\sqrt{3}\right],\\[2mm]
    0, & \text{иначе}.
  \end{cases}
\]
Матожидание: \( M(X)=0 \), дисперсия: \( D(X)=1 \).

\subsection{Статистики}

Пусть \( x_1,x_2,\dots,x_n \) --- выборка размера \( n \). Тогда:
\begin{itemize}
    \item Выборочное среднее:
    \[
      \bar{x} = \frac{1}{n} \sum_{i=1}^{n} x_i \ \ (1)
    \]
    \item Выборочная медиана (\(\mathrm{med}\)) --- центральный элемент упорядоченной выборки (либо среднее двух центральных при чётном \( n \)) \ \ (2)
    \item Полусумма квартилей (25\% и 75\%):
    \[
      z_Q = \frac{z_{1/4} + z_{3/4}}{2}, \ \ (3)
    \]
    где \( z_{1/4} \) --- 25\%-квантиль, \( z_{3/4} \) --- 75\%-квантиль.
\end{itemize}

По результатам 10, 50 и 1000 повторных экспериментов для каждой статистики вычисляются её среднее значение и дисперсия:
\[
E(z) = \langle z \rangle, \quad D(z) = \langle (z - E(z))^2 \rangle \ \ (4).
\]


\section{\Large Данные}
В данной таблице представленные все необходимые данные для каждого распределения и каждой выборки:
\begin{table}[h!]
\centering
\begin{tabular}{|c|c|c|c|c|c|c|c|c|}
\toprule
\hline
\ & Размер выборки & $\bar{x} \ (1)$ & $medx \ (2)$ & $z_Q \ (3)$ & $\bar{x}^2$ & $medx^2$ & $z_Q^2$ & $D(x) \ (4)$ \\
\midrule
\hline
\multirow{Cauchy} & 10 & 3.70 & 0.02 & -0.01 & $2\cdot10^4$ & 0.32 & 0.91 & $2\cdot10^4$ \\
 & 50 & 7.82 & 0.01 & 0.01 & $3.4\cdot10^4$ & 0.05 & 0.10 & $3.3\cdot10^4$ \\
 & 1000 & 1.20 & 0.00 & 0.00 & $1.3\cdot10^3$ & 0.00 & 0.00 & $1.3\cdot10^3$ \\
\midrule
\hline
\multirow{Normal} & 10 & 0.01 & 0.00 & 0.02 & 0.10 & 0.14 & 0.12 & 0.10 \\
 & 50 & -0.00 & -0.01 & -0.00 & 0.02 & 0.03 & 0.03 & 0.02 \\
 & 1000 & 0.00 & 0.00 & 0.00 & 0.00 & 0.00 & 0.00 & 0.00 \\
\midrule
\hline
\multirow{Poisson} & 10 & 10.02 & 9.87 & 9.93 & 101.44 & 98.87 & 99.86 & 0.99 \\
 & 50 & 9.98 & 9.82 & 9.89 & 99.87 & 96.73 & 98.07 & 0.19 \\
 & 1000 & 10.00 & 10.00 & 10.00 & 100.01 & 99.97 & 99.93 & 0.01 \\
\midrule
\hline
\multirow{Uniform} & 10 & 0.01 & 0.02 & 0.01 & 0.10 & 0.22 & 0.14 & 0.10 \\
 & 50 & 0.00 & 0.01 & 0.00 & 0.02 & 0.05 & 0.03 & 0.02 \\
 & 1000 & -0.00 & -0.00 & -0.00 & 0.00 & 0.00 & 0.00 & 0.00 \\
\bottomrule
\hline
\end{tabular}
\caption{Статистические характеристики для различных распределений и размеров выборок.}
\label{tab:stats}
\end{table}


\section{\Large Графики}

\begin{center}
    \includegraphics[scale=0.6]{1_Cauchy.png} \\
    Рис. 2 График распределения Коши
\end{center}

\begin{center}
    \includegraphics[scale=0.6]{1_normal.png} \\
    Рис. 3 График нормального распределения
\end{center}

\begin{center}
    \includegraphics[scale=0.6]{1_poisson.png} \\
    Рис. 4 График распределения Пуассона
\end{center}

\begin{center}
    \includegraphics[scale=0.6]{1_uniform.png} \\
    Рис. 5 График равномерного распределения
\end{center}

\section{\Large Вывод}
По полученным данным наглядно видно, что распределение Коши не подчиняется центральной предельной теореме - с увеличением размера выборки дисперсия растет. При этом проблемы возникают именно из-за среднего квадрата - квадрат среднего не сильно влияет. Для остальных распределений четко видно, что дисперсия убывает с ростом размера выборки.


\end{document}
